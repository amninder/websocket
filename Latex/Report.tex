\documentclass[12pt, titlepage, onecolumn]{article}

\usepackage{amsmath}
\usepackage[ruled, vlined]{algorithm2e}
\usepackage{graphicx}
\usepackage{color}
\usepackage{fixltx2e}
\usepackage[utf8]{inputenc}
\usepackage[T1]{fontenc}
\usepackage{hyperref}
\usepackage[margin=0.5in]
{geometry}
\usepackage{algorithmicx}
%\usepackage{algorithm, algorithmic}
\usepackage{algpseudocode}

% url link
\usepackage{hyperref}


% for using graphics
\usepackage{graphicx}
\geometry{a4paper}

% Minted
\usepackage{listings}
\usepackage{enumerate}
\usepackage{array}
\usepackage{url}
\include{pythonlisting}
\usepackage{minted}

\makeatletter
\newcommand\thefontsize[1]{}
\makeatother

%python highlight

\lstloadlanguages{Python}
\lstset{language=Python,tabsize=3}


\usepackage[francais, british]{babel}


\setlength\topmargin{0in}
\setlength\headheight{0in}
\setlength\headsep{0.2in}
\setlength\oddsidemargin{0in}
\setlength\textwidth{6.5in}
\setlength\textheight{9.0in}
\setlength\parindent{0.25in}
\setlength\parskip{0.25in}





\title{TechOps-Websocket}

\author{
Kevin Daum\thanks{Project Lead}\\
\small Application Developer\\[-0.8ex]
\small TechOps, Mount Pleasant, MI, USA\\[-0.8ex]
\small \texttt{daum1kc@cmich.edu}\\
\and
Amninder S Narota\thanks{Project Manager/Developer}\\
\small Developer\\[-0.8ex]
\small TechOps, Mount Pleasant, MI, USA\\[-0.8ex]
\small \texttt{narot1a@cmich.edu}
}
%\date{}


\begin{document}
\maketitle


%The body starts here
\section{Websocket}
This project is not intended to replace AJAX and is not strictly even a replacement for Comet/Long-poll (Although there are many cases where this makes sense).
\subsection{Requirements}
Before starting, make sure the following requirements/dependencies are fulfilled:
\begin{minted}{bash}
	Twisted==14.0.0
	autobahn==0.9.0
	six==1.8.0
	wsgiref==0.1.2
	zope.interface==4.1.1
\end{minted}

\subsection{Commands}
\label{sec:commands}
Following are the commands to execute web socket file:

\begin{enumerate}
	\item Using Python:
\begin{minted}{bash}
python ws.py
\end{minted}
	\item Using Twistd: 
	\label{item:twistd}
\begin{minted}{bash}
twistd -y ws.py

\end{minted}

\end{enumerate}
{\b Item \ref{item:twistd}} of {\b Section \ref{sec:commands}} is the production level command which will generate two files "\emph{twistd.log}" \& "\emph{twistd.pid}". "\emph{twist.log}" will keep the log of the the running process and PId is the unique ID assigned to each process running on the process. One possible command to kill a process with PID 4235 would be:
\begin{minted}{bash}
	kill -INT 4235
\end{minted}
or
\begin{minted}{bash}
	sudo kill -INT 4235
\end{minted}
which ever suits the situation.

\section{ws.py}
\label{sec:ws.py}
There are two ports which the server listens to:



\begin{enumerate}
	\item {\bf 8000}: When the Twisted Server (**Click here $\rightarrow$ \href{https://twistedmatrix.com/trac/}{\bf Twisted Network Programming Essentials\cite{Fettig:2005:TNP:1205685}}) is running the server listens to port 8000 as defined in following code but following code is to set port number for web socket process:
	
	\item{\bf 5000}: Websocket listens for change on port 5000.
	\label{item:5000}
\end{enumerate}
\inputminted[firstline=1, lastline=20, linenos=true]{python}{../ws.py}
\inputminted[firstline=21, lastline=42, linenos=true]{python}{../ws.py}

\section{broadcast.py}
This file consists of two classes which is imported in \emph{ws.py} as explained in \emph{\bf Section \ref{sec:ws.py}}

\subsection{BroadcastServerProtocol(WebSocketServerProtocol)}
This class is responsible to communicating network peers. \emph{\bf onMessage(self, payload, isBinary)} is the method which is listening to the port defined in \emph{\bf ws.py} of List Item \ref{item:5000} of Section \ref{sec:ws.py} and broadcasts the message to the network peers.

\begin{listing}[H]
	\inputminted[firstline=1, lastline=25]{python}{../broadcast/broadcast.py}
\end{listing}

\section{index.html}
Once the server is running and the message can be send through javascript websocket. Following is the sample html template for live chat over network:

\subsection{Html code Snippet}
\begin{listing}[H]
	\inputminted[firstline=77, lastline=99, linenos=true]{html}{../index.html}
\end{listing}

\subsection{JS code Snippet}
	\inputminted[firstline=3, lastline=76, linenos=true]{html}{../index.html}

\bibliography{ref}{}
\bibliographystyle{plain}
	
\end{document}